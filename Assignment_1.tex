% MAT 352 Assignment
% Author: 2018/2019 Computer Science
% Date: March 30, 2023

\documentclass[a4paper]{article}

\usepackage[margin=1.0in]{geometry}
\usepackage{amsmath}
\usepackage[table]{xcolor}
\usepackage{longtable}
\usepackage{fancyhdr}
\usepackage[auto-lang=false]{lipsum}
\usepackage{booktabs}

\setlength{\arrayrulewidth}{0.5mm}
\setlength{\tabcolsep}{18pt}
\renewcommand{\arraystretch}{1.5}

\begin{document}

    \begin{titlepage}

        \begin{center}
            \vspace*{1cm}

            \Huge
            \textbf{MAT 352 Assignment --- 1}

            \vspace{1.5cm}

            \textbf{Computer Science Department}

            \vspace{2cm}
            \normalsize
            \raggedright{
                \section*{Question:}
                Proof of the Inclusion-Exclusion Rule
            }

            \vspace{5cm}
            \centering
            Submitted to Dr.\ Adinya

            \vspace{1cm}
            \today
        \end{center}

    \end{titlepage}

    \pagenumbering{arabic}
	\pagestyle{fancy}
	\fancyhead{}
	\fancyhead[L]{\textbf{MAT 352 Assignment}}
	\fancyhead[R]{\textbf{Computer Science}}

    \section*{Name of Students}
    \begin{center}
        \rowcolors{4}{gray}{lightgray}
        \Large
        \begin{longtable} { c|l|c }
            \toprule[2pt]
            S/N & \multicolumn{1}{|c|}{Name} & Matric Number \\
            \midrule
            1 & Adebowale Joseph Akintomiwa & 214846\\
            2 & Adedapo Anjorin & 214864\\
            3 & Adegbola Olatunde Williams & 207186\\
            4 & Adeleke Sherifdeen Adeboye & 214848\\
            5 & Adeleke Timothy Toluwani & 214849\\
            6 & Adelowo Samuel Damilare & 214850\\
            7 & Adeoti Warith Adetayo & 214851\\
            8 & Adim Chimaobi Solomon & 222455\\
            9 & Adisa Inioluwa Christiana & 214853\\
            10 & Ahmad Animasaun & 214863\\
            11 & Ajayi Prince Ayokunle & 215221\\
            12 & Akinade Faith Eniola & 222459\\
            13 & Akinrinola Akinfolarin & 205526\\
            14 & Akinrinola, Blessing Opemipo & 214857\\
            15 & Akinwusi Ifeoluwa & 214858\\
            16 & Alao Tawakalit Omowunmi & 222461\\
            17 & Alatise Oluwaseun Abraham & 214860\\
            18 & Arowolo Ayomide Stephen & 214865\\
            19 & Brai Daniel & 214868\\
            20 & Chinedu Promise Okafor & 213930\\
            21 & Daniel Emmanuel Oghenetega & 224870\\
            22 & Denedo Oghenetega & 214873\\
            23 & Emiade James & 214874\\
            24 & Farayola Joshua Olatunde & 214878\\
            25 & Godwin Daniel & 214871\\
            26 & Ibraheem Nuh Babatunde & 214879\\
            27 & Ikwuegbu Michael & 214881\\
            28 & Kareem Mustapha Babatunde & 214883\\
            29 & Kayode Peter Temitope & 208077\\
            30 & Kehinde Boluwatife Soyoye & 214916\\
            31 & Kip Charles Okechukwu Emeka & 215061\\
            32 & Matric  & Number\\
            33 & Kubiat Laura & 214884\\
            34 & Lawal Uchechukwu Adebayo & 214885\\
            35 & Matthews Victoria Olayide & 214886\\
            36 & Nwatu Chidinma Augustina & 214890\\
            37 & Odulate Oluwatobi Gabriel & 214893\\
            38 & Ogbolu Precious Chiamaka & 214894\\
            39 & Oghie Daniel O. & 214895\\
            40 & Ogunesan Rhoda Oluwatosin & 214897\\
            41 & Ogunyemi Temidayo Samuel & 214898\\
            42 & Ojewale Opeoluwa David & 214899\\
            43 & Okafor Lisa Chisom & 214901\\
            44 & Okoro Joshua Akachukwu & 214902\\
            45 & Okumagba Oghenerukevwe Miracle & 222498\\
            46 & Olagidi Joshua & 222500\\
            47 & Olalere Khadijat Titilayo & 222502\\
            48 & Olatunji Michael Oluwayemi & 214903\\
            49 & Olawale Eniola Emmanuel & 214904\\
            50 & Olorogun Ebikabowei Caleb & 214906\\
            51 & Oluwatade Iyanuoluwa & 214907\\
            52 & Oluwayelu Oluwanifise & 215257\\
            53 & Onasoga Oluwapelumi Idris & 214909\\
            54 & Oyekanmi Eniola & 214913\\
            55 & Sadiq Peter & 214914\\
            56 & Salami Lateefat Abimbola & 214915\\
            57 & Stephen Chidiebere Ivuelekwa & 214882\\
            58 & Toluwanimi Oluwabukunmi Osuolale & 214912\\
            59 & Ubaka Amazing-Grace Onyiyechukwu & 214918\\
            60 & Uchechukwu Ahunanya & 214854\\
            61 & Wisdom Oyor & 215206\\
            \bottomrule[4pt]
        \end{longtable}
        \normalsize
    \end{center}

    \newpage
    \section*{Proof of the Inclusion-Exclusion Rule}
    The Inclusion-Exclusion States that:

    \begin{equation} \label{i_e_rule}
        \begin{split}
            P(E_1 \cup E_2 \cup E_3 \cup \dots \cup E_n) = & \sum_{i = 1}^{n} P(E_i) - \sum_{i_1 < i_2} P(E_{i_1} \cap E_{i_2}) + \sum_{i_1 < i_2 < i_3} P(E_{i_1} \cap E_{i_2} \cap E_{i_3}) + \cdots + \\
            & {(-1)}^{r + 1} \sum_{i_1 < i_2 < \cdots < i_r} P(E_{i_1} \cap E_{i_2} \cap \cdots \cap E_{i_r}) + \cdots + \\
            & {(-1)}^{n} \sum_{i_1 < i_2 < \cdots < i_{n-1}} P(E_{i_1} \cap E_{i_2} \cap \cdots \cap E_{i_{n-1}}) + \\
            & {(-1)}^{n + 1} P(E_1 \cap E_2 \cap E_3 \cap \cdots \cap E_n)
        \end{split}
    \end{equation}

    where the summation ${(-1)}^{r + 1} \sum_{i_1 < i_2 < \cdots < i_r} P(E_{i_1} \cap E_{i_2} \cap \cdots \cap E_{i_r})$ is taken over all the $\binom{n}{r}$ possible combinations.

    \subsection*{Proof by Principle of Mathematical Induction}

    Proof by PMI involves two steps,the \textbf{Basic Step} where the rule is shown for the base value/unit and the \textbf{Inductive step} which also involve two steps:
    \begin{enumerate}
        \item {Making an assumption that the rule/relation holds for an arbitrary value $n = k$}
        \item {Using this assumption to prove that the rule/relation holds for the ${(n = k + 1)}^{th}$ value.}
    \end{enumerate}

    \subsection{The Basic Step}
    For the Induction-Exclusion Rule, the most basic value (base case) is when the number of events is $2$ (i.e $n = 2$), therefore, the rule is applied as:
    \begin{equation} \label{base_case}
        P(E_1 \cup E_2) = P(E_1) + P(E_2) - P(E_1 \cap E_2)
    \end{equation}
    \subsection{The Inductive Step}
    \subsubsection*{The assumption that it the rule holds for some $n = k$}
    It is assumed that the rule/relation holds for a $n = k$ number of events. Therefore the equation below is noted.

    \begin{equation} \label{kth_rule}
        \begin{split}
            P(E_1 \cup E_2 \cup E_3 \cup \dots \cup E_k) = & \sum_{i = 1}^{k} P(E_i) - \sum_{i_1 < i_2} P(E_{i_1} \cap E_{i_2}) + \sum_{i_1 < i_2 < i_3} P(E_{i_1} \cap E_{i_2} \cap E_{i_3}) + \cdots + \\
            & {(-1)}^{r + 1} \sum_{i_1 < i_2 < \cdots < i_r} P(E_{i_1} \cap E_{i_2} \cap \cdots \cap E_{i_r}) + \cdots + \\
            & {(-1)}^{k} \sum_{i_1 < i_2 < \cdots < i_{k-1}} P(E_{i_1} \cap E_{i_2} \cap \cdots \cap E_{i_{k-1}}) + \\
            & {(-1)}^{k + 1} P(E_1 \cap E_2 \cap E_3 \cap \cdots \cap E_k)
        \end{split}
    \end{equation}

    \subsubsection*{The proof that the rule holds for $n = k + 1$}
    Now, the rule is also applied for $n = k + 1$ number of events:
    \begin{equation} \label{kplus_oneth_rule}
        \begin{split}
            P(E_1 \cup E_2 \cup E_3 \cup \dots \cup E_k \cup E_{k + 1}) = & \sum_{i = 1}^{k + 1} P(E_i) - \sum_{i_1 < i_2} P(E_{i_1} \cap E_{i_2}) + \sum_{i_1 < i_2 < i_3} P(E_{i_1} \cap E_{i_2} \cap E_{i_3}) \\
            & + \cdots + {(-1)}^{r + 1} \sum_{i_1 < i_2 < \cdots < i_r} P(E_{i_1} \cap E_{i_2} \cap \cdots \cap E_{i_r}) \\
            & + \cdots + {(-1)}^{k + 1} \sum_{i_1 < i_2 < \cdots < i_{k}} P(E_{i_1} \cap E_{i_2} \cap \cdots \cap E_{i_k}) \\
            & + {(-1)}^{k + 2} P(E_1 \cap E_2 \cap E_3 \cap \cdots \cap E_k \cap E_{k + 1})
        \end{split}
    \end{equation}

    Consider that $P(E_1 \cup E_2 \cup E_3 \cup \dots \cup E_k \cup E_{k + 1})$ can be written as $P([E_1 \cup E_2 \cup E_3 \cup \dots \cup E_k] \cup E_{k + 1})$ wherer by $``E_1 \cup E_2 \cup E_3 \cup \dots \cup E_k``$ is taken as a single event.
    \begin{equation} \label{split_to_base}
        \begin{split}
            P(E_1 \cup E_2 \cup E_3 \cup \dots \cup E_k \cup E_{k + 1}) = P([E_1 \cup E_2 \cup E_3 \cup \dots \cup E_k] \cup E_{k + 1})
        \end{split}
    \end{equation}

    By applying the Base Relation of the Inclusion-Exclusion Rule as in equation (\ref{base_case}) to the RHS of the equation (\ref{split_to_base}) above and comparing with equation (\ref{kplus_oneth_rule}) the rule/relation can be proven.

    \paragraph*{}So now:

    \begin{equation} \label{base_kplus_one}
        \begin{split}
            P([E_1 \cup E_2 \cup E_3 \cup \dots \cup E_k] \cup E_{k + 1}) = & P(E_1 \cup E_2 \cup E_3 \cup \dots \cup E_k) + P(E_{k + 1}) \\
            & - P([E_1 \cup E_2 \cup E_3 \cup \dots \cup E_k] \cap E_{k + 1})
        \end{split}
    \end{equation}

    The last term in equation (\ref{base_kplus_one}) above can be expanded  as:

    \begin{equation} \label{equ_l}
        \begin{split}
            P([E_1 \cup E_2 \cup E_3 \cup \dots \cup E_k] \cap E_{k + 1}) = P([E_1 \cap E_{k + 1}] \cup [E_2 \cap E_{k + 1}] \cup \cdots \cup [E_k \cap E_{k + 1}])
        \end{split}
    \end{equation}

    The RHS of equation (\ref{equ_l}) above can be further simplified by applying the Inclusion-Exclusion here again
    noting that $|P([E_1 \cap E_{k + 1}] \cup [E_2 \cap E_{k + 1}] \cup \cdots \cup [E_k \cap E_{k + 1}])| = k$

    \begin{equation} \label{equ_o}
        \begin{split}
            P([E_1 \cap E_{k + 1}] &\cup [E_2 \cap E_{k + 1}] \cup \cdots \cup [E_k \cap E_{k + 1}]) = \sum_{i = 1}^{k} P(E_i \cap E_{k + 1}) \\
            & - \sum_{i_1 < i_2} P([E_{i_1} \cap E_{k + 1}] \cap [E_{i_2} \cap E_{k + 1}]) \\
            & + \cdots + {(-1)}^{r} \sum_{i_1 < i_2 \cdots i_{r - 1}} P([E_{i_1} \cap E_{k + 1}] \cap [E_{i_2} \cap E_{k + 1}] \cap \cdots \cap [E_{i_{r - 1}} \cap E_{k + 1}]) \\
            & + \cdots + {(-1)}^{k} \sum_{i_1 < i_2 \cdots i_{k - 1}} P([E_{i_1} \cap E_{k + 1}] \cap [E_{i_2} \cap E_{k + 1}] \cap \cdots \cap [E_{i_{k - 1}} \cap E_{k + 1}]) \\
            & + {(-1)}^{k + 1} P([E_1 \cap E_{k + 1}] \cap [E_2 \cap E_{k + 1}] \cap \cdots \cap [E_k \cap E_{k + 1}])
        \end{split}
    \end{equation}

    From Set Theory:
    \begin{equation*}
        \begin{split}
            (A \cap B) \cap C = & A \cap B \cap C \\
            (A \cap B) \cap (A \cap C) = & A \cap B \cap C
        \end{split}
    \end{equation*}

    Some term in equation (\ref{equ_o}) can be rewritten as:

    \begin{equation*}
        \begin{split}
            &P([E_{i_1} \cap E_{k + 1}] \cap [E_{i_2} \cap E_{k + 1}]) = P(E_{i_1} \cap E_{i_2} \cap E_{k + 1}) \\
            &P([E_{i_1} \cap E_{k + 1}] \cap [E_{i_2} \cap E_{k + 1}] \cap \cdots \cap [E_{i_{r - 1}} \cap E_{k + 1}]) = P(E_{i_1} \cap E_{i_2} \cap \cdots \cap E_{i_{r - 1}} \cap E_{k + 1}) \\
            &P([E_{i_1} \cap E_{k + 1}] \cap [E_{i_2} \cap E_{k + 1}] \cap \cdots \cap [E_{i_{k - 1}} \cap E_{k + 1}]) = P(E_{i_1} \cap E_{i_2} \cap \cdots \cap E_{i_{k - 1}} \cap E_{k + 1}) \\
            &P([E_1 \cap E_{k + 1}] \cap [E_2 \cap E_{k + 1}] \cap \cdots \cap [E_k \cap E_{k + 1}]) = P(E_1 \cap E_2 \cap \cdots \cap E_k \cap E_{k + 1})
        \end{split}
    \end{equation*}

    Therefore equation (\ref{equ_o}) is simplified as:

    \begin{equation} \label{equ_p}
        \begin{split}
            P([E_1 \cap E_{k + 1}] &\cup [E_2 \cap E_{k + 1}] \cup \cdots \cup [E_k \cap E_{k + 1}]) = \sum_{i = 1}^{k} P(E_i \cap E_{k + 1}) \\
            & - \sum_{i_1 < i_2} P(E_{i_1} \cap E_{i_2} \cap E_{k + 1}) \\
            & + \cdots + {(-1)}^{r} \sum_{i_1 < i_2 \cdots i_{r - 1}} P(E_{i_1} \cap E_{i_2} \cap \cdots \cap E_{i_{r - 1}} \cap E_{k + 1}) \\
            & + \cdots + {(-1)}^{k} \sum_{i_1 < i_2 \cdots i_{k - 1}} P(E_{i_1} \cap E_{i_2} \cap \cdots \cap E_{i_{k - 1}} \cap E_{k + 1}) \\
            & + {(-1)}^{k + 1} P(E_1 \cap E_2 \cap \cdots \cap E_k \cap E_{k + 1})
        \end{split}
    \end{equation}

    By substituting equation (\ref{kth_rule}) and above equation (\ref{equ_p}) and in equation (\ref{base_kplus_one}), equation (\ref{base_kplus_one}) is then evaluated as:

    \begin{align}
        \begin{split}
            P([E_1 \cup E_2 &\cup E_3 \cup \dots \cup E_k] \cup E_{k + 1}) \\
            = & P(E_1 \cup E_2 \cup E_3 \cup \dots \cup E_k) + P(E_{k + 1}) - P([E_1 \cup E_2 \cup E_3 \cup \dots \cup E_k] \cap E_{k + 1})
        \end{split} \\
        \begin{split}
            = & \sum_{i = 1}^{k} P(E_i) - \sum_{i_1 < i_2} P(E_{i_1} \cap E_{i_2}) + \sum_{i_1 < i_2 < i_3} P(E_{i_1} \cap E_{i_2} \cap E_{i_3}) + \cdots + \\
            & {(-1)}^{r + 1} \sum_{i_1 < i_2 < \cdots < i_r} P(E_{i_1} \cap E_{i_2} \cap \cdots \cap E_{i_r}) + \cdots + \\
            & {(-1)}^{k} \sum_{i_1 < i_2 < \cdots < i_{k-1}} P(E_{i_1} \cap E_{i_2} \cap \cdots \cap E_{i_{k-1}}) + \\
            & {(-1)}^{k + 1} P(E_1 \cap E_2 \cap E_3 \cap \cdots \cap E_k) + P(E_{k + 1}) \\
            & - \left[ \sum_{i = 1}^{k} P(E_i \cap E_{k + 1}) - \sum_{i_1 < i_2} P(E_{i_1} \cap E_{i_2} \cap E_{k + 1}) \right. \\
            & \left. + \cdots + {(-1)}^{r} \sum_{i_1 < i_2 \cdots i_{r - 1}} P(E_{i_1} \cap E_{i_2} \cap \cdots \cap E_{i_{r - 1}} \cap E_{k + 1}) \right. \\
            & \left. + \cdots + {(-1)}^{k} \sum_{i_1 < i_2 \cdots i_{k - 1}} P(E_{i_1} \cap E_{i_2} \cap \cdots \cap E_{i_{k - 1}} \cap E_{k + 1}) \right. \\
            & \left. + {(-1)}^{k + 1} P(E_1 \cap E_2 \cap \cdots \cap E_k \cap E_{k + 1}) \right] \\
        \end{split}
    \end{align}
    Note that the term ``${(-1)}^{k} \sum_{i_1 < i_2 < \cdots < i_{k-1}} P(E_{i_1} \cap E_{i_2} \cap \cdots \cap E_{i_{k-1}})$`` can be contained in the terms ``${(-1)}^{r + 1} \sum_{i_1 < i_2 < \cdots < i_r} P(E_{i_1} \cap E_{i_2} \cap \cdots \cap E_{i_r}) + \cdots +$`` and therefore may be ignored. The equation is then reduced to:

    \begin{align}
        \begin{split} \label{reduced_term}
            P([E_1 \cup E_2 &\cup E_3 \cup \dots \cup E_k] \cup E_{k + 1}) \\
            = & \sum_{i = 1}^{k} P(E_i) - \sum_{i_1 < i_2} P(E_{i_1} \cap E_{i_2}) + \sum_{i_1 < i_2 < i_3} P(E_{i_1} \cap E_{i_2} \cap E_{i_3}) + \cdots + \\
            & {(-1)}^{r + 1} \sum_{i_1 < i_2 < \cdots < i_r} P(E_{i_1} \cap E_{i_2} \cap \cdots \cap E_{i_r}) + \cdots + \\
            & {(-1)}^{k + 1} P(E_1 \cap E_2 \cap E_3 \cap \cdots \cap E_k) + P(E_{k + 1}) \\
            & - \left[ \sum_{i = 1}^{k} P(E_i \cap E_{k + 1}) - \sum_{i_1 < i_2} P(E_{i_1} \cap E_{i_2} \cap E_{k + 1}) \right. \\
            & \left. + \cdots + {(-1)}^{r} \sum_{i_1 < i_2 \cdots i_{r - 1}} P(E_{i_1} \cap E_{i_2} \cap \cdots \cap E_{i_{r - 1}} \cap E_{k + 1}) \right. \\
            & \left. + \cdots + {(-1)}^{k} \sum_{i_1 < i_2 \cdots i_{k - 1}} P(E_{i_1} \cap E_{i_2} \cap \cdots \cap E_{i_{k - 1}} \cap E_{k + 1}) \right. \\
            & \left. + {(-1)}^{k + 1} P(E_1 \cap E_2 \cap \cdots \cap E_k \cap E_{k + 1}) \right] \\
        \end{split}
    \end{align}

    Expand the bracket in eqation (\ref{reduced_term}) above and distribute the negative sign noting that $-1 \times {(-1)}^{a + 1} = {(-1)}^{a + 2}$

    \begin{align}
        \begin{split} \label{expanded_term}
            P([E_1 \cup E_2 &\cup E_3 \cup \dots \cup E_k] \cup E_{k + 1}) \\
            = & \sum_{i = 1}^{k} P(E_i) - \sum_{i_1 < i_2} P(E_{i_1} \cap E_{i_2}) + \sum_{i_1 < i_2 < i_3} P(E_{i_1} \cap E_{i_2} \cap E_{i_3}) + \cdots + \\
            & {(-1)}^{r + 1} \sum_{i_1 < i_2 < \cdots < i_r} P(E_{i_1} \cap E_{i_2} \cap \cdots \cap E_{i_r}) + \cdots + \\
            & {(-1)}^{k + 1} P(E_1 \cap E_2 \cap E_3 \cap \cdots \cap E_k) + P(E_{k + 1}) \\
            & - \sum_{i = 1}^{k} P(E_i \cap E_{k + 1}) + \sum_{i_1 < i_2} P(E_{i_1} \cap E_{i_2} \cap E_{k + 1}) \\
            & + \cdots + {(-1)}^{r + 1} \sum_{i_1 < i_2 \cdots i_{r - 1}} P(E_{i_1} \cap E_{i_2} \cap \cdots \cap E_{i_{r - 1}} \cap E_{k + 1}) \\
            & + \cdots + {(-1)}^{k + 1} \sum_{i_1 < i_2 \cdots i_{k - 1}} P(E_{i_1} \cap E_{i_2} \cap \cdots \cap E_{i_{k - 1}} \cap E_{k + 1}) \\
            & + {(-1)}^{k + 2} P(E_1 \cap E_2 \cap \cdots \cap E_k \cap E_{k + 1}) \\
        \end{split}
    \end{align}

    Collecting like terms:

    \begin{align}
        \begin{split} \label{grouped_term}
            P([E_1 \cup E_2 &\cup E_3 \cup \dots \cup E_k] \cup E_{k + 1}) \\
            = & \sum_{i = 1}^{k} P(E_i) + P(E_{k + 1}) \\
            & - \sum_{i_1 < i_2} P(E_{i_1} \cap E_{i_2}) - \sum_{i = 1}^{k} P(E_i \cap E_{k + 1}) \\
            & + \sum_{i_1 < i_2 < i_3} P(E_{i_1} \cap E_{i_2} \cap E_{i_3}) + \sum_{i_1 < i_2} P(E_{i_1} \cap E_{i_2} \cap E_{k + 1}) + \cdots + \\
            & {(-1)}^{r + 1} \sum_{i_1 < i_2 < \cdots < i_r} P(E_{i_1} \cap E_{i_2} \cap \cdots \cap E_{i_r}) + \\
            & {(-1)}^{r + 1} \sum_{i_1 < i_2 \cdots i_{r - 1}} P(E_{i_1} \cap E_{i_2} \cap \cdots \cap E_{i_{r - 1}} \cap E_{k + 1}) + \cdots + \\
            & {(-1)}^{k + 1} P(E_1 \cap E_2 \cap E_3 \cap \cdots \cap E_k) + \\
            & {(-1)}^{k + 1} \sum_{i_1 < i_2 \cdots i_{k - 1}} P(E_{i_1} \cap E_{i_2} \cap \cdots \cap E_{i_{k - 1}} \cap E_{k + 1}) + \\
            & {(-1)}^{k + 2} P(E_1 \cap E_2 \cap \cdots \cap E_k \cap E_{k + 1}) \\
        \end{split}
    \end{align}

    Considering the collected like terms in equation (\ref{grouped_term}) above, the terms can be collapsed as follows (including summation bounds):

    \begin{align} \label{collapsed_term1}
        \sum_{i = 1}^{k} P(E_i) + P(E_{k + 1}) &= \sum_{i = 1}^{k + 1} P(E_i) \\
        \sum_{1 \le i_1 < i_2 \le k} P(E_{i_1} \cap E_{i_2}) + \sum_{i = 1}^{k} P(E_i \cap E_{k + 1}) &= \sum_{1 \le i_1 < i_2 \le {k + 1}} P(E_{i_1} \cap E_{i_2}) \\
        \sum_{1 \le i_1 < i_2 < i_3 \le k} P(E_{i_1} \cap E_{i_2} \cap E_{i_3}) + \sum_{1 \le i_1 < i_2 \le k} P(E_i \cap E_{k + 1}) &= \sum_{1 \le i_1 < i_2 < i_3 \le {k + 1}} P(E_{i_1} \cap E_{i_2} \cap E_{i_3})
    \end{align}

    \begin{multline} \label{collapsed_term2}
        \sum_{1 \le i_1 < i_2 < \cdots < i_{r \le k} \le k} P(E_{i_1} \cap E_{i_2} \cap \cdots \cap E_{i_r}) + \sum_{1 \le i_1 < i_2 \cdots i_{r - 1 \le k} \le k} P(E_{i_1} \cap E_{i_2} \cap \cdots \cap E_{i_{r - 1}} \cap E_{k + 1}) \\
        = \sum_{1 \le i_1 < i_2 \cdots i_{r \le {k + 1}} \le {k + 1}} P(E_{i_1} \cap E_{i_2} \cap \cdots \cap E_{i_r})
    \end{multline}

    \begin{multline} \label{collapsed_term3}
        P(E_{i_1} \cap E_{i_2} \cap \cdots \cap E_{i_k}) + \sum_{i_1 < i_2 \cdots i_{k - 1} \le k} P(E_{i_1} \cap E_{i_2} \cap \cdots \cap E_{i_{k - 1}} \cap E_{k + 1}) \\
        = \sum_{i_1 < i_2 \cdots < i_k \le {k + 1}} P(E_{i_1} \cap E_{i_2} \cap \cdots \cap E_{i_k})
    \end{multline}

    Substituting the collapsed equations (\ref{collapsed_term1}) to (\ref{collapsed_term3}) in equation (\ref{grouped_term}):

    \begin{align}
        \begin{split} \label{final_eq}
            P([E_1 \cup E_2 \cup E_3 \cup \dots \cup E_k] \cup E_{k + 1}) = & \sum_{i = 1}^{k + 1} P(E_i) - \sum_{1 \le i_1 < i_2 \le {k + 1}} P(E_{i_1} \cap E_{i_2}) \\
            & + \sum_{1 \le i_1 < i_2 < i_3 \le {k + 1}} P(E_{i_1} \cap E_{i_2} \cap E_{i_3}) \\
            & + \cdots + {(-1)}^{r + 1} \sum_{1 \le i_1 < i_2 \cdots i_{r \le {k + 1}} \le {k + 1}} P(E_{i_1} \cap E_{i_2} \cap \cdots \cap E_{i_r}) \\
            & + \cdots + {(-1)}^{k + 1} \sum_{i_1 < i_2 \cdots < i_k \le {k + 1}} P(E_{i_1} \cap E_{i_2} \cap \cdots \cap E_{i_k}) \\
            & + {(-1)}^{k + 2} P(E_1 \cap E_2 \cap \cdots \cap E_k \cap E_{k + 1}) \\
        \end{split}
    \end{align}

    Equation (\ref{final_eq}) above is equivalent to equation (\ref{kplus_oneth_rule}) (for the $n = k + 1$ number of events) and this concludes the proof for the Inclusion-Exclusion Rule.

    \subsection*{Closing Statement}
    With the proof shown above, it can be said that for any $n$ number of events, the Inclusion-Exclusion Rule:
    \begin{equation*}
        \begin{split}
            P(E_1 \cup E_2 \cup E_3 \cup \dots \cup E_n) = & \sum_{i = 1}^{n} P(E_i) - \sum_{i_1 < i_2} P(E_{i_1} \cap E_{i_2}) + \sum_{i_1 < i_2 < i_3} P(E_{i_1} \cap E_{i_2} \cap E_{i_3}) + \cdots + \\
            & {(-1)}^{r + 1} \sum_{i_1 < i_2 < \cdots < i_r} P(E_{i_1} \cap E_{i_2} \cap \cdots \cap E_{i_r}) + \cdots + \\
            & {(-1)}^{n} \sum_{i_1 < i_2 < \cdots < i_{n-1}} P(E_{i_1} \cap E_{i_2} \cap \cdots \cap E_{i_{n-1}}) + \\
            & {(-1)}^{n + 1} P(E_1 \cap E_2 \cap E_3 \cap \cdots \cap E_n)
        \end{split}
    \end{equation*}
    is true!!!


\end{document}
